\section{Silicon Pad Sensor Prototypes for the CMS Endcap Calorimeter Upgrade}
\label{sec:sensors}

The CMS high granularity calorimeter will be made of more than \SI{600}{\metre\squared} of planar DC-coupled silicon pad sensors.
The sensors are fabricated as 8'' hexagonal wafers to optimise the use of the circular wafer as which the silicon crystals are grown.
Due the empiric evidence of superior noise performance with respect to n-type sensors~\cite{Adam_2017}, p-type doping of their bulk was chosen.
Each HGCAL silicon sensor is segmented into several hundred pads that constitute the sensitive units. 
The majority of those pads are shaped as symmetric hexagons whereas special non-hexagonal structures fill out the wafer periphery.
All pads on a wafer are enclosed and protected from external disturbance by a guard ring.\newline
In this work, full-hexagonal wafer prototype sensors of the so-called low-density (LD) and high-density (HD) designs were irradiated with neutrons and electrically qualified.
Their design is illustrated in \textcolor{red}{add figure}.
The LD sensors were produced from physically thinned p-type float zone silicon wafers.
They are segmented into 198 pads, where full hexagonal pads are about \SI{1.2}{\centi\metre\squared} large, and have an active thickness of \SI{200}{\micro\meter} or \SI{300}{\micro\meter}.
It was found empirically~\cite{hgcal-tdr:2018} that the relative signal degradation is worse the thicker the active area of the silicon sensor is.
Therfore, LD sensors will be installed in regions of intermediate radiation fluences inside HGCAL.
By contrast, regions with the highest fluences will be populated with \SI{120}{\micro\meter}-thick HD sensors.
Those are segmented into 444 pads (full hexagon pad size: \SI{0.5}{\centi\metre\squared}) and are produced from epitaxial on a handle wafer.
Although their active thickness amounts to only \SI{120}{\micro\meter} HD sensors are physically \SI{300}{\micro\metre} thick.\newline

%check p-stop here: https://cds.cern.ch/record/2102890/files/CR2015_286.pdf
%and here: https://indico.cern.ch/event/818783/contributions/3598470/attachments/1950746/3238333/20191126_CHEF_final.pdf
Continue with mention of p-stop, flatband voltage, ,same oxide type - not relevant for this study

\begin{itemize}
    \item Explain different production parameters
    \item Define expected performance after irradiation
\end{itemize}