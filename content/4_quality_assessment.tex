\section{Sensor Quality Assessment}
\label{sec:QA}

\subsection{Full-Wafer Leakage Current}
\label{subsec:QA_Itot}

\begin{figure}
	\captionsetup[subfigure]{aboveskip=-1pt,belowskip=-1pt}
	\centering
	\begin{subfigure}[b]{0.49\textwidth}
		\includegraphics[width=0.999\textwidth]{plots/total_iv/total_current_IV.pdf}
		\subcaption{
		}
		\label{plot:tot_IV_good}
	\end{subfigure}
	\hfill
	\begin{subfigure}[b]{0.49\textwidth}
		\includegraphics[width=0.999\textwidth]{plots/total_iv/total_current_IV_bad.pdf}
		\subcaption{
		}
		\label{plot:tot_IV_bad}
	\end{subfigure}
	\caption{
		Full-wafer leakage currents at different effective bias voltages (a) for six good sensors from all irradiation rounds, and (b) for bad sensors with sudden leakage current increase.
	}
\end{figure}


\subsection{Per-Pad Leakage Currents}
\label{subsec:QA_Ipad}
\begin{figure}
	\captionsetup[subfigure]{aboveskip=-1pt,belowskip=-1pt}
	\centering
	\begin{subfigure}[b]{0.32\textwidth}
		\includegraphics[width=0.999\textwidth]{plots/iv_hexplots/3003.pdf}
		\subcaption{
		}
		\label{plot:iv_hexplot_3003}
	\end{subfigure}
	\hfill
	\begin{subfigure}[b]{0.32\textwidth}
		\includegraphics[width=0.999\textwidth]{plots/iv_hexplots/3009.pdf}
		\subcaption{
		}
		\label{plot:iv_hexplot_3009}
	\end{subfigure}
	\hfill
	\begin{subfigure}[b]{0.32\textwidth}
		\includegraphics[width=0.999\textwidth]{plots/iv_hexplots/0541_04.pdf}
		\subcaption{
		}
		\label{plot:iv_hexplot_0541_04}
	\end{subfigure}
	\hfill	
	\begin{subfigure}[b]{0.32\textwidth}
		\includegraphics[width=0.999\textwidth]{plots/iv_hexplots/2004.pdf}
		\subcaption{
		}
		\label{plot:iv_hexplot_2004}
	\end{subfigure}
	\hfill
	\begin{subfigure}[b]{0.32\textwidth}
		\includegraphics[width=0.999\textwidth]{plots/iv_hexplots/1013.pdf}
		\subcaption{
		}
		\label{plot:iv_hexplot_1013}
	\end{subfigure}
	\hfill
	\begin{subfigure}[b]{0.32\textwidth}
		\includegraphics[width=0.999\textwidth]{plots/iv_hexplots/1002.pdf}
		\subcaption{
		}
		\label{plot:iv_hexplot_1002}
	\end{subfigure}	
	\caption{
		Per-pad leakage currents interpolated to an effective bias voltage of \SI{600}{\volt} for six representative sensors from all irradiation rounds.
		The chuck temperature profile is corrected for, cf.~\ref{appendix:chuck_temp}.
		Note the different leakage current scales.
		Red- or white-colored edge pads correspond to well-understood (however undesired) measurement peculiarities, e.g. unconnected pogo pins.
	}
\end{figure}


\begin{figure}
	\captionsetup[subfigure]{aboveskip=-1pt,belowskip=-1pt}
	\centering
	\begin{subfigure}[b]{0.49\textwidth}
		\includegraphics[width=0.999\textwidth]{plots/channel_iv/channel_IV_sensors_sensors.pdf}
		\subcaption{
		}
		\label{plot:pad_IV_sensor}
	\end{subfigure}
	\hfill
	\begin{subfigure}[b]{0.49\textwidth}
		\includegraphics[width=0.999\textwidth]{plots/channel_iv/channel_IV_sensors_channels.pdf}
		\subcaption{
		}
		\label{plot:pad_IV_channels}
	\end{subfigure}
	\caption{
		Per-pad leakage currents at different effective bias voltages (a) for one central pad on six good sensors from all irradiation rounds, and (b) for different pads with different geometries on one example sensor.
	}
\end{figure}



\subsection{Per-Pad Capacitance and Depletion Voltage}
\label{subsec:QA_Vdep}

\begin{figure}
	\captionsetup[subfigure]{aboveskip=-1pt,belowskip=-1pt}
	\centering
	\begin{subfigure}[b]{0.49\textwidth}
		\includegraphics[width=0.999\textwidth]{plots/channel_cv/channel_CV_sensors_sensors.pdf}
		\subcaption{
		}
		\label{plot:pad_CV_sensor}
	\end{subfigure}
	\hfill
	\begin{subfigure}[b]{0.49\textwidth}
		\includegraphics[width=0.999\textwidth]{plots/channel_cv/channel_CV_sensors_channels.pdf}
		\subcaption{
		}
		\label{plot:pad_CV_channels}
	\end{subfigure}
	\caption{
		Per-pad capacitances at different effective bias voltages (a) for one central pad on six good sensors from all irradiation rounds, and (b) for different pads with different geometries on one example sensor.
		The LCR frequency in these measurements was \SI{2}{\kilo\hertz}.
	}
\end{figure}

\begin{figure}
	\captionsetup[subfigure]{aboveskip=-1pt,belowskip=-1pt}
	\centering
	\begin{subfigure}[b]{0.49\textwidth}
		\includegraphics[width=0.999\textwidth]{plots/channel_cv/channel_invCV_sensors_sensors.pdf}
		\subcaption{
		}
		\label{plot:pad_invCV_sensor}
	\end{subfigure}
	\hfill
	\begin{subfigure}[b]{0.49\textwidth}
		\includegraphics[width=0.999\textwidth]{plots/channel_cv/channel_invCV_sensors_channels.pdf}
		\subcaption{
		}
		\label{plot:pad_invCV_channels}
	\end{subfigure}
	\caption{
		Per-pad inverse capacitances at different effective bias voltages for estimate of the sensor depletion voltage (a) for one central pad on six good sensors from all irradiation rounds, and (b) for different pads with different geometries on one example sensor.
		The depletion voltage is known to be affected by the LCR frequency, which was set to \SI{2}{\kilo\hertz} in these measurements.
	}
\end{figure}


\begin{figure}
	\captionsetup[subfigure]{aboveskip=-1pt,belowskip=-1pt}
	\centering
	\begin{subfigure}[b]{0.49\textwidth}
		\centering
		\includegraphics[width=0.7\textwidth]{plots/Vdep_hexplots/0541_04.pdf}
		\subcaption{
			}
			\label{plot:Vdep_hexplot_0541_04}
	\end{subfigure}
	\hfill
	\begin{subfigure}[b]{0.49\textwidth}
		\centering
		\includegraphics[width=0.7\textwidth]{plots/Vdep_hexplots/1002.pdf}
		\subcaption{
		}
		\label{plot:Vdep_hexplot_1002}
	\end{subfigure}	
	\caption{
		Per-pad depletion voltage estimates for (a) ..., (b). 
		Fluence profile visible, cf.~\ref{plot:iv_hexplot_0541_04}, and~\ref{plot:iv_hexplot_1002}.
		Note the different depletion voltage scales.
	}
\end{figure}