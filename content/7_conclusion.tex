\section{Conclusion}
\label{sec:conclusion}
The CMS High Granularity Calorimeter (HGCAL) is the new proposed granular sampling calorimeter which will replace the existing CMS endcap calorimeters for High-Luminosity LHC.
For its compactness and its radiation hardness, p-doped silicon will be deployed as sensitive material in the regions with highest anticipated fluences.
The full silicon sensors are fabricated as 8'' hexagonal wafers with 198 (low density) or 444 (high density design) individual readout pads.
They will have been exposed to fluences ranging from  a few $10^{14}$ to about $10^{16}$ 1-MeV neutron equivalents per square-centimeter at the end of their lifetime.\newline
An essential component of the sensors' prototyping is the experimental verification of their radiation hardness in terms of their electrical properties, i.e. leakage currents, capacitances and depletion voltages.
The Rhode Island Neutron Irradiation Facility (RINSC) is one of the few locations world-wide that offers the infrastructure for the irradiation of large structures such as the HGCAL 8'' silicon wafers.
Given that parts of the necessary infrastructure have not been used in 30$~$years, the irradiation itself was non-trivial and specific preparations had to be made.
After irradiation, the probe- and switch card-based ARRAY system~\cite{pitters:array2019} was used for the first time for the electrical characterisation of neutron-irradiated silicon sensors at CERN and at Texas Tech University.
In order to protect the test system from currents that would exceed its specifications, the neutron-irradiated sensors were tested at \SI{-40}{\celsius}.\newline
This paper explained the preparation and conduction of the neutron-irradiation at RINSC and the electrical characterisation of the irradiated sensors at cold temperatures using the ARRAY system.
Evidence for non-trivial properties of the RINSC irradiation facility such as a non-uniform flux profile across the 8'' wafer and for non-negligible sensor annealing due to insufficient cooling during irradiation was reported.
We consider the procedural documentation on the neutron-irradiation and on the subsequent electrical characterisation relevant input for future R$\&$D on silicon-based calorimeter concepts for potential future colliders.\newline
namely
%overall radiation hardness as expected
%process parameters have no impact
%extra annealing has beneficial impact on currents and depletion voltage
%despite prototype sensors: reference for exptected degradation of HGCAL's silicon sensors towards the end of its lifetime

