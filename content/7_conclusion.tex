\section{Conclusion}
\label{sec:conclusion}
The CMS High Granularity Calorimeter (HGCAL) is the new proposed granular sampling calorimeter which will replace the existing CMS endcap calorimeters for High-Luminosity LHC (HL-LHC).
For its compactness, its intrinsic timing resolution, and its radiation hardness, p-doped silicon will be deployed as sensitive material in the regions with highest anticipated fluences.
The full silicon sensors are fabricated as 8'' hexagonal wafers with 198 (low density) or 444 (high density design) individual readout pads.
They will have been exposed to fluences ranging from  a few $10^{14}$ to about $10^{16}~\neqcm$ towards the end of HL-LHC's 10-year operation.\newline
An essential component of the sensors' prototyping is the experimental verification of their radiation hardness in terms of their electrical properties, i.e. leakage currents, capacitances and depletion voltages.
The Rhode Island Nuclear Science Center (RINSC) is one of the few locations world-wide that offers the infrastructure for the irradiation of large structures such as HGCAL's 8''-wide silicon pad sensors.
Given that parts of the necessary infrastructure have not been used in this experimental capacity, the irradiation itself was non-trivial and specific preparations had to be made.
After irradiation, the probe- and switch card-based ARRAY system~\cite{pitters:array2019} was used for the first time for the electrical characterisation of neutron-irradiated silicon sensors at CERN and at Texas Tech University.
In order to protect the test system from currents that would exceed its specifications, the neutron-irradiated sensors were cooled down to \SI{-40}{\celsius} during the tests.\newline
This paper explained the preparation and execution of the neutron-irradiation at RINSC and the electrical characterisation of the irradiated sensors at cold temperatures using the ARRAY system.
Evidence for non-trivial properties of the RINSC irradiation facility such as a non-uniform flux profile across the 8'' wafer and for non-negligible sensor annealing due to insufficient cooling during irradiation was reported.
We consider the procedural documentation on the neutron-irradiation and on the subsequent electrical characterisation relevant input for future R$\&$D on silicon-based detector concepts for potential future colliders.\newline
For the planned CMS HGCAL upgrade, the findings in this work are encouraging as they confirm the overall expected radiation hardness of the prototype silicon sensors.
In particular, their leakage current densities are found to scale proportionally with the fluence, independent on the properties of the tested production process variations.
We reconfirm that annealing the sensors up to \SI{80}{\minute} at \SI{60}{\celsius} has beneficial impact on the electrical properties by lowering the dark current and the depletion voltages by a few 10$~\%$.
Despite the tested sensors were prototypes, the insights from their characterisation can be considered representative for the final version.
Therefore, the results reported in this work may serve as reference for the expected electrical performance and degradation of HGCAL's silicon sensors towards the end of their lifetime at HL-LHC.