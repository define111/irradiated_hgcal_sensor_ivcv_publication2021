\section{Results}
\label{sec:results}

\subsection{Leakage Current}
\label{subsec:leakagecurrents}

\begin{figure}
	\captionsetup[subfigure]{aboveskip=-1pt,belowskip=-1pt}
	\centering
	\begin{subfigure}[b]{0.49\textwidth}
		\includegraphics[width=0.999\textwidth]{plots/total_iv/total_current_IV.pdf}
		\subcaption{
		}
		\label{plot:tot_IV_good}
    \end{subfigure}
    \hfill
    \begin{subfigure}[b]{0.49\textwidth}
        \includegraphics[width=0.999\textwidth]{plots/channel_iv/channel_IV_sensors_channels.pdf}
        \subcaption{
        }
        \label{plot:pad_IV_channels}
    \end{subfigure}

	\caption{
		(a) Two representative full-wafer leakage currents after irradiation (without additional annealing). Measurements were taken at \SI{-40}{\celsius} ($\text{I}_\text{tot, \SI{-40}{\celsius}}$) and for different effective bias voltages ($\text{U}_\text{bias}$). 
        (b) Per-pad leakage currents as a function of the effective bias voltage for different pads with different geometries on one example sensor normalised to the area of full hexagonal pads.
	}
\end{figure}


\begin{figure}
	\captionsetup[subfigure]{aboveskip=-1pt,belowskip=-1pt}
	\centering
	\begin{subfigure}[b]{0.49\textwidth}
		\includegraphics[width=0.999\textwidth]{plots/annealing_iv/annealing_IV_ch24.pdf}
		\subcaption{
		}
		\label{plot:annealing_IV}
	\end{subfigure}
	\hfill
	\begin{subfigure}[b]{0.49\textwidth}
		\includegraphics[width=0.999\textwidth]{plots/annealing_iv/annealing_current.pdf}
		\subcaption{
		}
		\label{plot:annealing_current}
	\end{subfigure}    

	\caption{
		(a) IV-curves of a representative full hexagonal pad for different annealing scenarios for a \SI{200}{\micro\metre} low-density prototype sensor irradiated to approximately 2.4$~$E15 1-MeV-neutron equivalents/cm$^{2}$.
        (b) Decrease of the per-pad leakage current (interpolated to $U_\text{bias}=U_\text{dep}$) as a function of the additional annealing time at \SI{60}{\celsius} for a subset of full hexagonal pads.
	}
\end{figure}


\begin{figure}
	\captionsetup[subfigure]{aboveskip=-1pt,belowskip=-1pt}
	\centering
	\begin{subfigure}[b]{0.32\textwidth}
		\includegraphics[width=0.999\textwidth]{plots/iv_hexplots/3009.pdf}
		\subcaption{
		}
		\label{plot:iv_hexplot_3009}
	\end{subfigure}
	\hfill
	\begin{subfigure}[b]{0.32\textwidth}
		\includegraphics[width=0.999\textwidth]{plots/iv_hexplots/0541_04.pdf}
		\subcaption{
		}
		\label{plot:iv_hexplot_0541_04}
	\end{subfigure}
	\hfill	
	\begin{subfigure}[b]{0.32\textwidth}
		\includegraphics[width=0.999\textwidth]{plots/iv_hexplots/1013.pdf}
		\subcaption{
		}
		\label{plot:iv_hexplot_1013}
	\end{subfigure}
    \hfill
	\begin{subfigure}[b]{0.32\textwidth}
		\includegraphics[width=0.999\textwidth]{plots/iv_hexplots/3009_annealed.pdf}
		\subcaption{
		}
		\label{plot:iv_hexplot_3009_annealed}
	\end{subfigure}
	\hfill
	\begin{subfigure}[b]{0.32\textwidth}
		\includegraphics[width=0.999\textwidth]{plots/iv_hexplots/0541_04_annealed.pdf}
		\subcaption{
		}
		\label{plot:iv_hexplot_0541_04_annealed}
	\end{subfigure}
	\hfill	
	\begin{subfigure}[b]{0.32\textwidth}
		\includegraphics[width=0.999\textwidth]{plots/iv_hexplots/1013_annealed.pdf}
		\subcaption{
		}
		\label{plot:iv_hexplot_1013_annealed}
	\end{subfigure}    
	\label{plot:iv_hexplot}
	\caption{
		Per-pad leakage currents interpolated to an effective bias voltage of \SI{600}{\volt} for three representative sensors from different irradiation rounds before (a-c) and after additional annealing (d-e).
		The chuck temperature profile is corrected for, cf.~\ref{appendix:chuck_temp}.
		Red- or white-colored edge pads correspond to well-understood (however undesired) measurement peculiarities, e.g. unconnected pogo pins.
		Note the different leakage current colour scales.
	}
\end{figure}


\begin{figure}
	\captionsetup[subfigure]{aboveskip=-1pt,belowskip=-1pt}
	\centering
    \includegraphics[width=0.69\textwidth]{plots/alpha/alpha_600V.pdf}
    \label{plot:alpha_600}
	\caption{
	    Volume-normalised per-pad leakage current for different fluences at a bias voltage of \SI{600}{\volt}.
        Prototype sensors were characterised after additional annnealing at \SI{60}{\celsius} at CERN and at Texas Tech University (TTU).
		Measured leakage currents are scaled to a room-temperature of \SI{+20}{\celsius}.
        The current-relate damage rate ($\alpha$) is found to be independent of the sensor production parameters investigated in this work.
	}
\end{figure}


\subsection{Capacitance and Depletion Voltage}
\label{subsec:Udep}


\begin{figure}
	\captionsetup[subfigure]{aboveskip=-1pt,belowskip=-1pt}
	\centering

	\begin{subfigure}[b]{0.49\textwidth}
		\includegraphics[width=0.999\textwidth]{plots/annealing_Vdep/annealing_CV_ch24.pdf}
		
		\subcaption{
		}
        \label{plot:annealing_CV}
	\end{subfigure}
    \hfill
    \begin{subfigure}[b]{0.49\textwidth}
		\includegraphics[width=0.999\textwidth]{plots/annealing_Vdep/annealing_Vdep.pdf}
		\subcaption{
		}		
        \label{plot:annealing_Vdep}
	\end{subfigure}
	\caption{
        (a) Inverted CV-curves of a representative full hexagonal pad for different annealing scenarios for a \SI{200}{\micro\metre} low-density prototype sensor irradiated to approximately 2.4$~$E15 1-MeV-neutron equivalents/cm$^{2}$.   
		(b) Relative decrease of the depletion voltage estimate ($U_\text{dep}$) as a function of the additional annealing time at \SI{60}{\celsius} for a subset of full pads.
	}
\end{figure}

\begin{figure}
	\captionsetup[subfigure]{aboveskip=-1pt,belowskip=-1pt}
	\centering
	\begin{subfigure}[b]{0.49\textwidth}
		\includegraphics[width=0.999\textwidth]{plots/channel_cv/channel_CV_sensors_channels.pdf}
		\subcaption{
		}
		\label{plot:pad_CV_channels}
	\end{subfigure}
	\hfill
	\begin{subfigure}[b]{0.49\textwidth}
		\includegraphics[width=0.999\textwidth]{plots/channel_cv/channel_invCV_sensors_sensors.pdf}
		\subcaption{
		}
		\label{plot:pad_invCV_sensor}
	\end{subfigure}
	\caption{
		(a) Area-normalised capacitances as a function of the effective bias voltage for different pads with different geometries on one example sensor.
		(b) Normalised squared-inverse capacitances as a function of the effective bias voltage for estimating the sensor depletion voltage for one central pad on three example sensors from different irradiation rounds.
		The LCR frequency in these measurements was \SI{2}{\kilo\hertz}.
	}
\end{figure}



\begin{figure}
	\captionsetup[subfigure]{aboveskip=-1pt,belowskip=-1pt}
	\centering
	\begin{subfigure}[b]{0.49\textwidth}
		\centering
		\includegraphics[width=0.99\textwidth]{plots/Vdep_hexplots/0541_04.pdf}
		\subcaption{
			}
			\label{plot:Vdep_hexplot_0541_04}
	\end{subfigure}
	\hfill
	\begin{subfigure}[b]{0.49\textwidth}
		\centering
		\includegraphics[width=0.999\textwidth]{plots/Vdep_vs_fluence/Vdep_vs_current_5414.pdf}
		\subcaption{
			}
			\label{plot:Vdep_vs_current_5414}
	\end{subfigure}
	\caption{
		(a) Per-pad depletion voltage estimates for a \SI{200}{\micron} LD example sensor irradiated to 1.9E+15$~$neq/cm$^{2}$, and 
		(b) their correlation to the per-pad leakage current, as proxy for the fluence.
	}
\end{figure}

%todo: vdep hexplot after annealing, vdep vs fluence