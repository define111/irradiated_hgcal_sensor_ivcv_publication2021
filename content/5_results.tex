\section{Results}
\label{sec:results}

\begin{itemize}
	\item Irradiation worsens electrical properties of silicon sensors.
	\item Bulk-dominated leakage current per volumne expected to increase proportionally with fluence, investigated in~\ref{subsec:irradiation_alpha}.
	\item Depletion voltage expected to increase, shown in~\ref{subsec:irradiation_Vdep}.		
	\item Characterisation measurements discussed here are before after additional annealing
	\item Annealing, i.e. heating up, during long shutdowns
	\item \SI{80}{\minute} at \SI{60}{\celsius} corresponds to xx at yy or cc at bb (best-case scenario)
	\item Characterisation measurements discussed here performed at \SI{-40}{\celsius} and at very low humidity
	\item Overall: Characteristics as expected after irradiation, and after annealing
	\item Show pecularity of irradiation of large scale objects
\end{itemize}

\subsection{Leakage Current}
\label{subsec:leakagecurrents}
\begin{figure}
	\captionsetup[subfigure]{aboveskip=-1pt,belowskip=-1pt}
	\centering
	\begin{subfigure}[b]{0.49\textwidth}
		\includegraphics[width=0.999\textwidth]{plots/total_iv/total_current_IV.pdf}
		\subcaption{
		}
		\label{plot:tot_IV_good}
    \end{subfigure}
    \hfill
    \begin{subfigure}[b]{0.49\textwidth}
        \includegraphics[width=0.999\textwidth]{plots/channel_iv/channel_IV_sensors_channels.pdf}
        \subcaption{
        }
        \label{plot:pad_IV_channels}
    \end{subfigure}

	\caption{
		(a) Total leakage currents after irradiation (without additional annealing) for two representative example sensors. 
		Currents were measured at \SI{-40}{\celsius} ($\text{I}_\text{tot, \SI{-40}{\celsius}}$) and at different effective bias voltages ($\text{U}_\text{bias}$). 
        (b) Per-pad leakage currents normalised to the area of full hexagonal pads as a function of the effective bias voltage for different pads with different geometries on one example sensor.
	}
\end{figure}
The total leakage current is defined as the current flowing through R$_\text{HV}$ in \ref{fig:switchprobecard_CAD} and can be understood as the dark current of a full silicon wafer.
As it is exemplified for two prototype sensors in \ref{plot:tot_IV_good}, the total leakage current did not exhibit exponential growth and stayed well below the ARRAY system compliance of \SI{2}{\milli\ampere} for most of the irradiated sensors that were tested up to \SI{850}{\volt} at \SI{-40}{\celsius}.
Irreversible discharges are undesired but in fact occurred for a handful of the tested sensors where the total leakage current during electrical characterisation suddenly increased and exceeded the \SI{2}{\milli\ampere} limitation.
Whereas one half of those instances could be traced back to pre-existing conditions such as mechanical damages induced during the irradiation or the transport, the other half lead to a design modification of the HGCAL silicon sensors which should minimise the risk of discharges in the future\footnote{More than 20 prototype sensors with the improved design have been characterized with voltages up to \SI{850}{\volt} in the meantime. None have shown discharges thus far.}.
Results of the affected sensors are not discussed further in this work.
\ref{plot:pad_IV_channels} shows the per-pad leakage current as a function of the effective bias voltage (\emph{IV}) for adjacent pads on one representative sensor irradiated to intermediate fluences.
In general, our data demonstrate that the leakage current of a pad after irradiation scales with its volume, and that the relative increase from \SI{600}{\volt} to \SI{800}{\volt} remains well below \SI{150}{\percent}.
The beneficial effect of annealing on the leakage current has been studied in more detail for a sensor which was not exposed to high temperatures and with it was not affected by significant annealing during irradiation.
The measured IV curves after different annealing times are shown for a representative pad in \ref{plot:annealing_IV}.
\begin{figure}
	\captionsetup[subfigure]{aboveskip=-1pt,belowskip=-1pt}
	\centering
	\begin{subfigure}[b]{0.49\textwidth}
		\includegraphics[width=0.999\textwidth]{plots/annealing_iv/annealing_IV_ch24.pdf}
		\subcaption{
		}
		\label{plot:annealing_IV}
	\end{subfigure}
	\hfill
	\begin{subfigure}[b]{0.49\textwidth}
		\includegraphics[width=0.999\textwidth]{plots/annealing_iv/annealing_current.pdf}
		\subcaption{
		}
		\label{plot:annealing_current}
	\end{subfigure}    

	\caption{
		(a) IV-curves of a representative full hexagonal pad for different annealing scenarios for a \SI{200}{\micro\metre} low-density prototype sensor irradiated to approximately 2.4$~$E15 1-MeV-neutron equivalents/cm$^{2}$.
        (b) Decrease of the per-pad leakage current (interpolated to $U_\text{bias}=\SI{600}{\volt}$) as a function of the annealing time at \SI{60}{\celsius} for a subset of full hexagonal pads that are uniformly distributed over the full wafer.
	}
\end{figure}
For bias voltages beyond full depletion, per-pad leakage currents at \SI{600}{\volt} are reduced systematically by \SI{25}{\percent} after \SI{80}{\minute} at \SI{60}{\celsius}, cf. \ref{plot:annealing_current}.
Given the simultaneous reduction in the depletion voltage, cf. \ref{plot:annealing_CV}, leakage currents around full depletion are even reduced further, namely by \SI{35}{\percent} (not shown).
After annealing all sensors to an equivalent of \SI{80}{\minute} at \SI{60}{\celsius}, we observe the expected proportionality of the per-pad leakage current density to the anticipated fluence that the sensors was exposed to.
The proportionality for current densities interpolated to an effective bias voltage of \SI{600}{\volt}, which is well above full depletion for the investigated sensors, and extrapolated to \SI{20}{\celsius} is displayed in \ref{plot:alpha_600}.
\begin{figure}
	\captionsetup[subfigure]{aboveskip=-1pt,belowskip=-1pt}
	\centering
    \includegraphics[width=0.69\textwidth]{plots/alpha/alpha_600V.pdf}
    \label{plot:alpha_600}
	\caption{
	    Volume-normalised per-pad leakage currents interpolated to an effective bias voltage of \SI{600}{\volt} after different irradiation fluences.
        Currents were measured after additional annealing at \SI{60}{\celsius} at CERN or at Texas Tech University (TTU).
		Measured leakage currents are scaled to a room-temperature of \SI{20}{\celsius}.
	}
\end{figure}
The current-related damage rate ($\alpha$) is found to be independent of the silicon material properties investigated in this work which is consistent with previous findings, e.g. Ref.~\cite{MOLL199987}.
However, the numerical result for $\alpha$ is about \SI{15}{\percent} smaller than the literature value~\cite{moll:SiDamages}, hereby potentially indicating a systematic overestimate of the fluence at the RINSC irradiation facility. 
Even after correction for the chuck temperature non-uniformity, per-pad leakage current densities across a sensor at a fixed voltage differ by up to \SI{20}{\percent}.
The associated current profiles are present both before (\ref{plot:iv_hexplot_3009,plot:iv_hexplot_0541_04,plot:iv_hexplot_1013}) and after additional annealing (\ref{plot:iv_hexplot_3009_annealed,plot:iv_hexplot_0541_04_annealed,plot:iv_hexplot_1013_annealed}), are continuous, and are identical for sensors that had been irradiated simultaneously in the same puck.
These findings are interpreted as evidence for the presence of a gaussian-like ($\sigma\approx\mathcal{O}(\SI{10}{\centi\metre})$) fluence profile within the beamport of the RINSC irradiation facility.
\begin{figure}
	\captionsetup[subfigure]{aboveskip=-1pt,belowskip=-1pt}
	\centering
	\begin{subfigure}[b]{0.32\textwidth}
		\includegraphics[width=0.999\textwidth]{plots/iv_hexplots/3009.pdf}
		\subcaption{
		}
		\label{plot:iv_hexplot_3009}
	\end{subfigure}
	\hfill
	\begin{subfigure}[b]{0.32\textwidth}
		\includegraphics[width=0.999\textwidth]{plots/iv_hexplots/0541_04.pdf}
		\subcaption{
		}
		\label{plot:iv_hexplot_0541_04}
	\end{subfigure}
	\hfill	
	\begin{subfigure}[b]{0.32\textwidth}
		\includegraphics[width=0.999\textwidth]{plots/iv_hexplots/1013.pdf}
		\subcaption{
		}
		\label{plot:iv_hexplot_1013}
	\end{subfigure}
    \hfill
	\begin{subfigure}[b]{0.32\textwidth}
		\includegraphics[width=0.999\textwidth]{plots/iv_hexplots/3009_annealed.pdf}
		\subcaption{
		}
		\label{plot:iv_hexplot_3009_annealed}
	\end{subfigure}
	\hfill
	\begin{subfigure}[b]{0.32\textwidth}
		\includegraphics[width=0.999\textwidth]{plots/iv_hexplots/0541_04_annealed.pdf}
		\subcaption{
		}
		\label{plot:iv_hexplot_0541_04_annealed}
	\end{subfigure}
	\hfill	
	\begin{subfigure}[b]{0.32\textwidth}
		\includegraphics[width=0.999\textwidth]{plots/iv_hexplots/1013_annealed.pdf}
		\subcaption{
		}
		\label{plot:iv_hexplot_1013_annealed}
	\end{subfigure}    
	\caption{
		Per-pad leakage currents interpolated to an effective bias voltage of \SI{600}{\volt} for three representative sensors from different irradiation rounds before (a-c) and after additional annealing (d-f).
		Red- or white-colored edge pads correspond to well-understood measurement effects, e.g. insufficient contact between the pogo pins and the pads.
		}
	\label{plot:iv_hexplot}
\end{figure}




\subsection{Capacitance and Depletion Voltage}
\label{subsec:Udep}

The measured per-channel complex impedances are open-corrected by subtracting the impedances of the ARRAY system with floating pins.
Subsequently, the sensor pad capacitances are computed assuming a serial circuitry model.
Due to the finite mobility of the sensor defects, the frequency at which the impedance is measured may have sizeable impact on the hereby derived capacitances and depletion voltages for irradiated silicon pad sensors, cf. Ref.~\cite{Li1991}.
The impedance measurement for the results presented in this section was conducted at an LCR-frequency ($f_\text{LCR}$) of \SI{2}{\kilo\hertz}.
Experimental follow-up studies indicate a \SI{2}{\percent} impact on the end-capacitance and a \SI{10}{\percent} reduction of the depletion voltage estimates when reducing $f_\text{LCR}$ to \SI{500}{\hertz}.
This capacitance and depletion voltage dependence on $f_\text{LCR}$ should not alter the interpretations presented in the following.
As illustrated by \ref{plot:annealing_CV}, the saturation of the squared reciprocal of the pad capacitance ($C^{-2}$) is shifted to lower bias voltages ($V$) the more annealing occurs. 
The depletion voltage in this work is estimated from the intersection of a straight-line fitted to the rising part of the $C^{-2}V$ curves and of a line fitted to its plateau.
Since such a plateau is not always reached within the tested bias voltage range for all sensors prior to annealing (see black line in \ref{plot:annealing_CV}), the discussion below is based on the sensors characterisations after annealing for \SI{80}{\min} at \SI{60}{\celsius}. 


\begin{figure}
	\captionsetup[subfigure]{aboveskip=-1pt,belowskip=-1pt}
	\centering

	\begin{subfigure}[b]{0.49\textwidth}
		\includegraphics[width=0.999\textwidth]{plots/annealing_Vdep/annealing_CV_ch24.pdf}
		\subcaption{
		}
        \label{plot:annealing_CV}
	\end{subfigure}
    \hfill
    \begin{subfigure}[b]{0.49\textwidth}
		\includegraphics[width=0.999\textwidth]{plots/annealing_Vdep/annealing_Vdep.pdf}
		\subcaption{
		}		
        \label{plot:annealing_Vdep}
	\end{subfigure}
	\caption{
        (a) Inverted CV-curves of a representative full hexagonal pad for different annealing scenarios for a \SI{200}{\micro\metre} low-density prototype sensor irradiated to approximately 2.4$~$E15 1-MeV-neutron equivalents/cm$^{2}$.   
		(b) Relative decrease of the depletion voltage estimate ($U_\text{dep}$) as a function of the additional annealing time at \SI{60}{\celsius} for a subset of full pads.
	}
\end{figure}


\begin{itemize}
	\item Capacitance dominated by bulk, area-normalised capacitance on one sensor = stable within 5 percent, interpad contributions can explain differences?, cf.~\ref{plot:pad_CV_channels}
	\item Depletion voltage, i.e. where final capacitance is reached, clearly differs between fluence scenarios and thicknesses, cf.~\ref{plot:pad_invCV_sensor}, provide table?
	\item Estimate depends on choice of voltage steps, limited time
	\item Mention inter-pad capacitance from TTU
	\item Depletion voltage hexplot shows profile, matches leakage current profile
\end{itemize}


\begin{figure}
	\captionsetup[subfigure]{aboveskip=-1pt,belowskip=-1pt}
	\centering
	\begin{subfigure}[b]{0.49\textwidth}
		\includegraphics[width=0.999\textwidth]{plots/channel_cv/channel_CV_sensors_channels.pdf}
		\subcaption{
		}
		\label{plot:pad_CV_channels}
	\end{subfigure}
	\hfill
	\begin{subfigure}[b]{0.49\textwidth}
		\includegraphics[width=0.999\textwidth]{plots/channel_cv/channel_invCV_sensors_sensors.pdf}
		\subcaption{
		}
		\label{plot:pad_invCV_sensor}
	\end{subfigure}
	\caption{
		(a) Area-normalised capacitances as a function of the effective bias voltage for different pads with different geometries on one example sensor.
		(b) Normalised squared-inverse capacitances as a function of the effective bias voltage for estimating the sensor depletion voltage for one central pad on three example sensors from different irradiation rounds.
		The LCR frequency in these measurements was \SI{2}{\kilo\hertz}.
	}
\end{figure}



\begin{figure}
	\captionsetup[subfigure]{aboveskip=-1pt,belowskip=-1pt}
	\centering
	\begin{subfigure}[b]{0.49\textwidth}
		\centering
		\includegraphics[width=0.99\textwidth]{plots/Vdep_hexplots/0541_04.pdf}
		\subcaption{
			}
			\label{plot:Vdep_hexplot_0541_04}
	\end{subfigure}
	\hfill
	\begin{subfigure}[b]{0.49\textwidth}
		\centering
		\includegraphics[width=0.999\textwidth]{plots/Vdep_vs_fluence/Vdep_vs_current_5414.pdf}
		\subcaption{
			}
			\label{plot:Vdep_vs_current_5414}
	\end{subfigure}
	\caption{
		(a) Per-pad depletion voltage estimates for a \SI{200}{\micron} LD example sensor irradiated to 1.9E+15$~$neq/cm$^{2}$, and 
		(b) their correlation to the per-pad leakage current, as proxy for the fluence.
	}
\end{figure}

\begin{itemize}
	\item Unlike leakage current: Depletion voltage vs. fluence do not agree between different sensors (as expected). Reason: sizable dependence on production parameters, e.g. active thickness
	\item To test impact of fluence on $U_\text{dep}$: Exploit leakage current profile (taken proportional to fluence profile) to demonstrate positive correlation of depletion voltage to fluence.
	\item Use per-pad leakage current at \SI{600}{\volt} as proxy for fluence
	\item In fact: Observe positive correlation, cf.~\ref{plot:Vdep_vs_current_5414}
	\item Proportionality constant depends on thickness but is consistent between sensors with same thicknesses
\end{itemize}