\section{Isothermal Annealing Studies}
\label{sec:annealing}

\begin{itemize}
	\item Annealing, i.e. heating up, during long shutdowns
	\item \SI{80}{\minute} at \SI{60}{\celsius} corresponds to xx at yy or cc at bb
	\item Systematic annealing study for all sensors was beyond the scope of this study, and not possible for low annealing times due to high temperatures / annealing during irradiation
	\item Hence: focus on one illustrative example
	\item \ref{plot:annealing_IV}: illustrate the improvement in leakage current with annealing (study a sensor with minimal additional annealing during irradiation)
	\item IV curves lowered especially at high bias voltages beyond full depletion where bulk currents dominate
	\item \ref{plot:annealing_CV} shows a comparison of inverted CV curves after different annealing times, earlier saturation clearly visible, no change in end capacitance (as expected)
	\item \ref{plot:annealing_Vdep}: Depletion voltage estimated (cf. \ref{subsec:QA_Vdep}) reduced sizeably, about 40$~\%$ in case of \SI{200}{\micron} after \SI{80}{\min}
	\item Note that current at \SI{600}{\volt} reduced by approximately 25$~\%$ for all pads after \SI{80}{\min}
	\item Benefitting from both leakage current decrease and depletion voltage decrease = \ref{plot:annealing_current}: Current at depletion voltage reduced by approximately 35$~\%$ for all pads
\end{itemize}

\begin{figure}
	\captionsetup[subfigure]{aboveskip=-1pt,belowskip=-1pt}
	\centering
	\begin{subfigure}[b]{0.49\textwidth}
		\includegraphics[width=0.999\textwidth]{plots/annealing_iv/annealing_IV_ch24.pdf}
		\subcaption{
		}
		\label{plot:annealing_IV}
	\end{subfigure}
	\hfill
	\begin{subfigure}[b]{0.49\textwidth}
		\includegraphics[width=0.999\textwidth]{plots/annealing_Vdep/annealing_CV_ch24.pdf}
		
		\subcaption{
		}
		\label{plot:annealing_current}
	\end{subfigure}
	\caption{
		(a) IV-curves of a representative full hexagonal pad for different annealing scenarios for a \SI{200}{\micro\metre} low-density prototype sensor irradiated to 2.5$~$E15 1-MeV-neutron equivalents/cm$^{2}$.
        (b) Inverted CV-curves of a representative full hexagonal pad for different annealing scenarios for a \SI{200}{\micro\metre} low-density prototype sensor irradiated to 2.5$~$E15 1-MeV-neutron equivalents/cm$^{2}$.
	}
\end{figure}

\begin{figure}
	\captionsetup[subfigure]{aboveskip=-1pt,belowskip=-1pt}
	\centering
	\begin{subfigure}[b]{0.49\textwidth}
		\includegraphics[width=0.999\textwidth]{plots/annealing_Vdep/annealing_Vdep.pdf}
		\subcaption{
		}
		\label{plot:annealing_CV}
	\end{subfigure}
	\hfill
	\begin{subfigure}[b]{0.49\textwidth}
		\includegraphics[width=0.999\textwidth]{plots/annealing_iv/annealing_current_atUdep.pdf}
		\subcaption{
		}
		\label{plot:annealing_Vdep}
	\end{subfigure}
	\caption{
		(a) Relative Decrease of the depletion voltage estimate ($U_\text{dep}$) as a function of the additional annealing time at \SI{60}{\celsius} for a subset of full pads.
        (b) Decrease of the per-pad leakage current (interpolated to $U_\text{bias}=U_\text{dep}$) as a function of the additional annealing time at \SI{60}{\celsius} for a subset of full pads.
	}
\end{figure}