\section{Introduction}
\label{sec:introduction}
In the middle of the 2020's, the Large Hadron Collider~\cite{evans:2008} (LHC) at CERN will be upgraded to the High-Luminosity LHC (HL-LHC)~\cite{hl-lhc-tdr:2017}.
Its instantaneous luminosity is designed to reach at least five times the LHC's design value.
The increase in both the expected number of pile-up interactions and the anticipated damage due to the enhanced radiation levels will pose significant challenges to the LHC experiments.
As preparation for a successful operation at the HL-LHC, the CMS~\cite{cms:2008} collaboration has undertaken an extensive R$\&$D program to upgrade many parts of the detector.
One of the planned upgrades consists of the replacement of the existing endcap calorimeters with a novel high-granularity calorimeter (HGCAL)~\cite{hgcal-tdr:2018}.
HGCAL will consist of 47 sampling layers interspersed with absorber plates and will feature close to six million readout channels.
For its compactness, fast signal formation and acceptable radiation hardness, silicon has been chosen as the main sensitive material.
Approximately $\SI{600}{\metre\squared}$ of it will be deployed in the regions that will have been exposed to integrated fluences between a few $10^{14}~$ to $10^{16}~$1-MeV neutron-equivalents per square centimeter ($\neqcm$) at the end of its 10-year operation at HL-LHC. 
Prototypes of silicon-based modules have already been built and were tested with particle beams~\cite{cms_hgc-2016-beamtests,H1:2020,H2:2020,H3:2021}, establishing the proof-of-concept of this ambitious device.
The ultimate silicon sensors will be fabricated as multi-pad, DC-coupled, p-type ("n-in-p"), 8'' hexagonal wafers with active thicknesses of \SI{120}{\micro\metre}, \SI{200}{\micro\metre} or \SI{300}{\micro\metre}.
To assess their anticipated degradation over the HL-LHC lifetime, irradiation studies are an integral component to HGCAL's R$\&$D efforts.
However, the irradiation and electrical characterisation of full 8'' silicon sensors is not trivial.
In fact, previous R$\&$D on the HGCAL silicon sensors' degradation in terms of their leakage currents and depletion voltages was, thus far, limited to testing of small, approximately \SI{1}{\centi\metre\squared}-sized, single-diode test structures~\cite{Curr_s_2017,Akchurin:2020}.
As part of HGCAL's prototyping phase in the last few years, the CMS collaboration has developed the required infrastructure and procedures for the irradiation and the electrical characterisation of its 8'' silicon pad sensors.
The irradiation with neutrons was performed at the Rhode Island Nuclear Science Center (RINSC) which offers the necessary infrastructure to expose such large objects to fluences up to  $\sim 10^{16}~\neqcm$ within an acceptable time frame of a few hours.
Afterwards, the ARRAY system~\cite{pitters:array2019} was used for the electrical characterisation of all $\mathcal{O}(100)$ pads on the silicon sensors.
In order to reduce the fluence-induced leakage current to measurable levels, the tests were conducted at \SI{-40}{\celsius}.
%Considering that the ARRAY system was originally designed for testing of non-irradiated silicon pad sensors at room temperature where pad currents are typically three orders of magnitude smaller, its successful application was not self-evident.
Ultimately, the results from the conducted measurements represent an important milestone towards the realisation of the HGCAL.
In particular, they may serve as practical indications on the expected degradation of its silicon sensors towards the end of their lifetime at HL-LHC. 
The hereby developed infrastructure and procedures that enabled the irradiation and testing of those large, multi-pad silicon sensors may be instructive for future R$\&$D on novel silicon-based detector concepts.\newline
This paper is structured as follows:
\ref{sec:sensors} deals with the tested HGCAL silicon pad sensor prototypes by describing their design and the production parameters.
\ref{sec:irradiation} is about the neutron irradiation of the 8'' wafers at RINSC. 
The electrical test setup based on the ARRAY system and the measurement procedure are documented in~\ref{sec:setup}.
Results on the electrical sensor characteristics and their limitations are discussed in~\ref{sec:results}.
Finally, the conclusions are given in~\ref{sec:conclusion}.