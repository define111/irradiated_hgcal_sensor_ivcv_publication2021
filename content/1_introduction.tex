\section{Introduction}
\label{sec:introduction}
In the middle of this decade, the Large Hadron Collider (LHC)~\cite{evans:2008} at CERN will be upgraded to the High Luminosity LHC~\cite{hl-lhc-tdr:2017} where the provided instantaneous luminosities will reach five times the LHC's design value.
Both the expected number of pile-up interactions and the anticipated detector damages due to the enhanced irradiation level pose challenges to the LHC experiments.\newline
As preparation for a successful operation at the HL-LHC, the CMS~\cite{cms:2008} collaboration has undertaken an extensive R$\&$D program to upgrade many parts of the detector.
One of the envisaged upgrades consists of the replacement of the existing CMS endcap calorimeters with a highly granular calorimeter design (HGCAL)~\cite{hgcal-tdr:2018}.
HGCAL will consist of 47 sampling layers interspersed with absorber plates and features close to six million readout channels.
Silicon has been chosen as sensitive material for its compactness, fast signal formation and acceptable radiation hardness.
Approximately $\SI{600}{\metre\squared}$ of it will be deployed in the regions that will have been exposed to integrated fluences between a few $10^{14}~$ to $10^{16}~$1-MeV neutron-equivalents per square centimeter ($\neqcm$) after the end of its 10-year operation. 

sensors produced as 8'' hexagonal wafers~\cite{cms_hgc-2016-beamtests,H1:2020,H2:2020,Curr_s_2017}

%beam tests = proof-of-principle
%radiation hardness = single diodes, previous R&D
%full-wafer irradiations = important project milestone
%this paper: irradiation of full8'' wafers
% tests with array system
% experimental insight on what to expect

% structure is as follows




