\section{Introduction}
\label{sec:introduction}
In the middle of this decade, the Large Hadron Collider~\cite{evans:2008} (LHC) at CERN will be upgraded to the High-Luminosity LHC~\cite{hl-lhc-tdr:2017} (HL-LHC) where the provided instantaneous luminosity is designed to reach at least five times the LHC's design value.
Both the expected number of pile-up interactions and the anticipated detector damages due to the enhanced irradiation levels pose challenges to the LHC experiments.\newline
As preparation for a successful operation at the HL-LHC, the CMS~\cite{cms:2008} collaboration has undertaken an extensive R$\&$D program to upgrade many parts of the detector.
One of the planned upgrades consists of the replacement of the existing endcap calorimeters with a highly granular calorimeter (HGCAL)~\cite{hgcal-tdr:2018}.
HGCAL will consist of 47 sampling layers interspersed with absorber plates and will feature close to six million readout channels.
For its compactness, fast signal formation and acceptable radiation hardness, silicon has been chosen as the main sensitive material.
Approximately $\SI{600}{\metre\squared}$ of it will be deployed in the regions that will have been exposed to integrated fluences between a few $10^{14}~$ to $10^{16}~$1-MeV neutron-equivalents per square centimeter ($\neqcm$) after the end of its 10-year operation at HL-LHC. 
The silicon sensors will be fabricated as p-doped, 8'' hexagonal wafers with active thicknesses of \SI{120}{\micro\metre}, \SI{200}{\micro\metre} or \SI{300}{\micro\metre}.\newline
Prototypes of silicon-based modules have already been built and were tested with particle beam~\cite{cms_hgc-2016-beamtests,H1:2020,H2:2020}, establishing the proof-of-concept of this ambitious calorimeter.
R$\&$D on the silicon sensors' degradation in terms of their leakage currents and depletion voltages was, thus far, limited to testing of small, single-diode test structures, see e.g. Ref.~\cite{Curr_s_2017}.
Both the irradiation and the electrical characterisation of full 8'' silicon wafers is not trivial: 
For the irradiation, the corresponding facility must offer the necessary infrastructure to expose such large objects to fluences up to $\approx 10^{16}~\neqcm$ within acceptable time.
Secondly, an efficient electrical characterisation of all $\mathcal{O}(100)$ pads on such a silicon wafer requires specialised test circuitry.\newline
As part of HGCAL's prototyping phase over the last years, the CMS collaboration has successfully developed the required test infrastructure and procedures.
In 2020/21, full silicon sensor prototypes could be neutron-irradiated to fluences corresponding to the end of its lifetime for the first time. 
The irradiation took place at the Rhode Island Neutron Irradiation Facility. 
Subsequently, the sensors were kept cool and shipped to CERN where they were electrically characterised at temperatures of \SI{-40}{\celsius}.
For this purpose, the usage of the ARRAY system~\cite{pitters:array2019} was essential whose applicability for the measurement of $\mu$A-currents could be demonstrated.
Ultimately, the insights from the measurements represent an important milestone towards the realisation of HGCAL.
In particular, the results may serve as practical indications on the expected degradation of its silicon sensors. 
In addition, the hereby developed infrastructure and procedures that enabled the irradiation and the testing of those large, multi-pad silicon wafers may be instructive for eventual R$\&$D for future silicon-based detector concepts.\newline
This paper is structured as follows.
\ref{sec:sensors} deals with the tested HGCAL silicon sensor prototypes describing their design and production parameters.
Special emphasis is put on the neutron irradiation of the large wafers. 
\ref{sec:irradiation} is devoted to it.
The electrical test setup based on the ARRAY system and the procedure are explained in~\ref{sec:setup}.
Results on the electrical characteristics and their limitations are discussed in~\ref{sec:results}.
Finally, the conclusions are given in~\ref{sec:conclusion}.