\documentclass[a4paper,11pt]{article}
\pdfoutput=1 

\usepackage{jinstpub} 
\usepackage[utf8]{inputenc}
\usepackage{lineno}

\usepackage[capitalise,noabbrev,nameinlink]{cleveref}
\newlength{\abc}
\settowidth{\abc}{\space}
\AtBeginDocument{%
\renewcommand{\ref}[1]{\mbox{\Cref{#1}}}
\renewcommand{\equationautorefname}{Equation~\hspace{-\abc}}
\renewcommand{\sectionautorefname}{Section\,\negthinspace}
\renewcommand{\subsectionautorefname}{Section\,\negthinspace}
\renewcommand{\subsubsectionautorefname}{Section\,\negthinspace}
\renewcommand{\figureautorefname}{Figure\,\negthinspace}
\renewcommand{\tableautorefname}{Table\,\negthinspace}
}

% si units
\usepackage[binary-units=true]{siunitx}

% multirows in tables
\usepackage{multirow}

\usepackage{textcomp}
\usepackage{xcolor}
\usepackage{graphicx}
\usepackage{mathtools}
\usepackage{subcaption}
\usepackage[LGRgreek]{mathastext}
\usepackage{xspace}
%
% text handling
%

% vertical offset shortcut for paragraphs
\newcommand{\pg}[0]{\vspace*{0.75em}\newline}

% indentation shortcut
\newcommand{\ind}[0]{\hspace*{\customindent}}

% combination of vertical offset and indentation
\newcommand{\parag}{\parag\ind}


%
% content shorthands
%

% text and name highlights
\newcommand{\thl}[1]{``#1''}
\newcommand{\nhl}[1]{\textsc{#1}}

% missing citations
\newcommand{\tocite}[1]{\textbf{\textcolor{red}{[\ifthenelse{\equal{#1}{}}{?}{#1}]}}}

% arXiv and cds link
\newcommand{\arxivlink}[2]{\href{https://arxiv.org/abs/#1}{\texttt{arXiv:#1\,[#2]}}}
\newcommand{\cdslink}[1]{\href{http://cds.cern.ch/record/#1}{\texttt{cds:#1}}}

% argmin/max operators
\DeclareMathOperator*{\argmin}{arg\ min}
\DeclareMathOperator*{\argmax}{arg\ max}

% uneven plus-minus value
\newcommand{\upm}[4]{#1#2/\!#3\!#4}

%special signs
\newcommand{\inches}{\ensuremath{{}^{\prime\prime}}}
\newcommand{\skiroc}{SKIROC2-CMS}
\newcommand{\percent}{$\%$}
\newcommand{\permille}{\ensuremath{{}^\text{o}\mkern-5mu/\mkern-3mu_\text{oo}}}
\newcommand{\med}{\text{med}}

%% Units
\newcommand{\degrees}{\ensuremath{^{\circ}}\xspace}
\newcommand{\hitspermmbx}{\ensuremath{\text{Hits/mm}^{2}\text{/BX}}\xspace}
\newcommand{\permmbx}{\ensuremath{\text{/mm}^{2}\text{/BX}}\xspace}
\newcommand{\flux}{\ensuremath{\text{n_{eq}}/\text{cm}^{2}/\text{yr}}\xspace}
\newcommand{\neqcm}{\ensuremath{\mathrm{n_{eq}}/\mathrm{cm}^{2}}\xspace}
\newcommand{\percms}{\ensuremath{\text{cm}^{-2}\mathrm{s}^{-1}}\xspace}
\newcommand{\micron}{\ensuremath{\upmu\mathrm{m}}\xspace}
\newcommand{\microsecond}{\ensuremath{\upmu\mathrm{s}}\xspace}
\newcommand{\microwatt}{\ensuremath{\upmu\mathrm{W}}\xspace}

%abbreviations
\newcommand{\eg}{e.g.}
\newcommand{\ie}{i.e.}

%software
\newcommand{\CMSSW}{\textsc{CMSSW}}
\newcommand{\ROOT}{\textsc{ROOT}}
\newcommand{\corryvreckan}{\textsc{CORRYVRECKAN}}
\newcommand{\Geant}{\textsc{GEANT4}}
\newcommand{\luigi}{\textsc{LUIGI}}
\newcommand{\millepede}{\textsc{MILLEPEDE}}

%
% physics processes
%

\newcommand{\slep}{\text{single-lepton}}
\newcommand{\dlep}{\text{dilepton}}
\newcommand{\fullh}{\text{full-hadron}}
\newcommand{\Slep}{\text{Single-lepton}}
\newcommand{\Dlep}{\text{Dilepton}}
\newcommand{\Fullh}{\text{Full-hadron}}
\newcommand{\slepe}{\text{single-electron}}
\newcommand{\slepmu}{\text{single-muon}}
\newcommand{\Slepe}{\text{Single-electron}}
\newcommand{\Slepmu}{\text{Single-muon}}

\newcommand{\hf}{\text{heavy-flavor}}
\newcommand{\ttbar}{\ensuremath{t\bar{t}}}
\newcommand{\bbbar}{\ensuremath{b\bar{b}}}
\newcommand{\ccbar}{\ensuremath{c\bar{c}}}
\newcommand{\totbar}{\ensuremath{t\hspace{-0.5mm}/\hspace{-0.5mm}\bar{t}}}
\newcommand{\bobbar}{\ensuremath{b\hspace{-0.5mm}/\hspace{-0.5mm}\bar{b}}}
\newcommand{\ttH}{\ensuremath{\ttbar H}}
\newcommand{\Hbb}{\ensuremath{H\rightarrow \bbbar}}
\newcommand{\Hnbb}{\ensuremath{H\nrightarrow \bbbar}}
\newcommand{\HWW}{\ensuremath{H\rightarrow W^+W^-}}
\newcommand{\Htautau}{\ensuremath{H\rightarrow \tau^+\tau^-}}
\newcommand{\ttHbb}{\ensuremath{\ttH\,(\Hbb)}}
\newcommand{\ttHnbb}{\ensuremath{\ttH\,(\Hnbb)}}
\newcommand{\ttX}{\ensuremath{\ttbar\!+\!\text{X}}}
\newcommand{\ttbb}{\ensuremath{\ttbar\!+\!\bbbar}}
\newcommand{\ttb}{\ensuremath{\ttbar\!+\!\bobbar}}
\newcommand{\tttb}{\ensuremath{\ttbar\!+\!2b}}
\newcommand{\ttcc}{\ensuremath{\ttbar\!+\!\ccbar}}
\newcommand{\ttlf}{\ensuremath{\ttbar\!+\!\text{lf}}}
\newcommand{\tthf}{\ensuremath{\ttbar\!+\!\text{hf}}}
\newcommand{\Singlet}{\ensuremath{\text{Single}\ \totbar}}
\newcommand{\singlet}{\ensuremath{\text{single}\ \totbar}}
\newcommand{\Singlett}{\ensuremath{\text{Single}\ t}}
\newcommand{\Singletbar}{\ensuremath{\text{Single}\ \bar{t}}}
\newcommand{\Vjets}{\ensuremath{V\hspace{-0.5mm}\!+\!\text{jets}}}
\newcommand{\Wjets}{\ensuremath{W\hspace{-0.5mm}\!+\!\text{jets}}}
\newcommand{\Zjets}{\ensuremath{Z\hspace{-0.5mm}\!+\!\text{jets}}}
\newcommand{\Diboson}{\ensuremath{\text{Diboson}}}
\newcommand{\diboson}{\ensuremath{\text{diboson}}}
\newcommand{\ttW}{\ensuremath{\ttbar\!+\!W}}
\newcommand{\ttZ}{\ensuremath{\ttbar\!+\!Z}}
\newcommand{\ttV}{\ensuremath{\ttbar\!+\!V}}


%
% physics process colors
%

\definecolor{ttHcol}{RGB}{67,118,201}
\definecolor{ttbbcol}{RGB}{235,230,10}
\definecolor{ttbcol}{RGB}{205,0,10}
\definecolor{tttbcol}{RGB}{255,153,1}
\definecolor{ttcccol}{RGB}{131,38,10}
\definecolor{ttlfcol}{RGB}{81,142,25}
\definecolor{Singletcol}{RGB}{0,204,204}
\definecolor{Vjetscol}{RGB}{104,140,140}
\definecolor{Dibosoncol}{RGB}{1,25,147}
\definecolor{ttVcol}{RGB}{255,102,101}


%
% other physics variables
%

\newcommand{\BR}{\ensuremath{B\!R}}
\newcommand{\met}{\ensuremath{\cancel{E}_T}}
\DeclareRobustCommand{\orderof}{\ensuremath{\mathcal{O}}}


\title{\boldmath Electrical characteristics of silicon pad sensor prototypes for the CMS endcap calorimeter upgrade}


\author[a]{Thorben Quast}
\author[a]{Eva Sicking}
\author[a]{many more ...}

\affiliation[a]{CERN EP-DT}

\collaboration[b]{on behalf of the CMS collaboration}

\emailAdd{thorben.quast@cern.ch}

\abstract{
Here will be an abstract that will blow your mind.
} 
\keywords{Calorimeter, HGCAL, silicon sensors, capacitance, leakage current, electrical characterisation, neutron irradiation, Hamburg model}

\linenumbers

\begin{document}
\maketitle
\flushbottom

\linenumbers
\section{Introduction}
\label{sec:introduction}
\textcolor{red}{Todo.}\newline

% The Large Hadron Collider at CERN will be upgraded to provide more instantaneous luminosity~\cite{hllhc-tdr:2017}.
% Experiments, such as CMS, need to upgrade their detector systems. 
%One of the CMS upgrades is the replacement of the current endcap calorimeters with the High Granularity Calorimeter (HGCAL)~\cite{hgcal-tdr:2018}.
% Silicon sensors will be used and their electrical properties are important to assess after irradiation.
% This is what we will show here.
% Exposing silicon sensors to radiation introduces defects in the bulk and with it worsens their electrical properties in view of their application as particle detectors.
% In particular, the bulk-dominated leakage current density is expected to increase proportionally with the fluence, and sensors are expected to deplete at higher bias voltages with increasing fluence levels.
% Annealing can cure radiation-induced defects to some extent, hereby reducing the degradation of the leakage current and the depletion voltage.
\section{Prototype Silicon Sensors for CMS Endcap Calorimeter Upgrade}
\label{sec:sensors}

\begin{itemize}
    \item Explain common parameters
    \item Explain different production parameters
    \item Define grading critera / expectations
\end{itemize}
\section{Electrical Characterisation Setup}
\label{sec:setup}

\subsection{Measurement Principle}
\label{subsec:setup_principle}
\begin{figure}[h]
	\centering
	\includegraphics[width=0.75\textwidth]{figures/circuit_cards_updated.png}
	\label{fig:switchprobecard_CAD}
	\caption{
		Circuitry of our tests, taken from~\cite{pitters:array2019}
	}
\end{figure}


\begin{itemize}
	\item ARRAY system~\cite{pitters:array2019}
	\item Total leakage current over \SI{12}{\kilo\ohm} resistor corresponds to drop of effective voltage which is corrected for
	\item Low frequency preferable to minimise influence on sensor, but associated to error minimal for 2kHz (SPICE simulation ARRAY paper), less than 10percent impact on depletion voltage, less than 2percent impact on final capacitance (cf. Philipp)
\end{itemize}

\begin{itemize}
	\item Cold station Wentworth + model
	\item Measurements conducted at low temperatures to prevent hight currents and to protect the system
\end{itemize}




\subsection{Measurement Procedure}
\label{subsec:setup_procedure}
\begin{itemize}
	\item 1st round: Per-pad leakage currents (IV), 2nd round: per-pad capacitance
	\item Most measurements at set chuck temperature of \SI{-40}{\celsius}
	\item Current compliances to protect system:
	\item Total current compliance: \SI{2}{\milli\ampere}
	\item Per-pad current compliance: \SI{5}{\micro\ampere}
	\item Cells in compliance masked in subsequent capacitance measurement
	\item LCR frequency for capacitance measurement is \SI{2}{\kilo\hertz}
	\item Open-correction, serial definition of capacitance
	\item Iterate over all channels at constant voltage, then switch voltage
	\item Total characterisation time: IV = \SI{90}{\minute} (LD)
	\item Total characterisation time: CV = \SI{150}{\minute} (LD)

\end{itemize}
\section{Sensor Quality Assessment}
\label{sec:QA}

\subsection{Full-Wafer Leakage Current}
\label{subsec:QA_Itot}

\begin{figure}[h]
	\centering
	\includegraphics[width=0.49\textwidth]{plots/total_iv/total_current_IV.pdf}
	\label{plot:tot_IV_good}
	\caption{
	Full-wafer leakage currents at different effective bias voltages for six sensors representing all irradiation rounds.
	}
\end{figure}

\subsection{Per-Pad Leakage Currents}
\label{subsec:QA_Ipad}

\subsection{Per-Pad Capacitance and Depletion Voltage}
\label{subsec:QA_Vdep}

\section{Isothermal Annealing Studies}
\label{sec:annealing}

\begin{figure}
	\captionsetup[subfigure]{aboveskip=-1pt,belowskip=-1pt}
	\centering
	\begin{subfigure}[b]{0.49\textwidth}
		\includegraphics[width=0.999\textwidth]{plots/annealing_iv/annealing_IV_ch24.pdf}
		\subcaption{
		}
		\label{plot:annealing_IV}
	\end{subfigure}
	\hfill
	\begin{subfigure}[b]{0.49\textwidth}
		\includegraphics[width=0.999\textwidth]{plots/annealing_iv/annealing_current.pdf}
		\subcaption{
		}
		\label{plot:annealing_current}
	\end{subfigure}
	\caption{
		(a) Representative per-pad IV-curves for different annealing scenarios for a \SI{200}{\micro\metre} low-density prototype sensor irradiated to 2.5$~$E15 1-MeV-neutron equivalents/cm$^{2}$.
        (b) Decrease of the per-pad leakage current (interpolated to $U_\text{bias}=\SI{600}{\volt}$) as a function of the additional annealing time at \SI{60}{\celsius}.
	}
\end{figure}
\section{Effect of Irradiation on Electrical Sensor Properties}
\label{sec:irradiation}

\subsection{Current-Related Damage Rate}
\label{subsec:irradiation_alpha}

\begin{figure}
	\captionsetup[subfigure]{aboveskip=-1pt,belowskip=-1pt}
	\begin{subfigure}[b]{0.49\textwidth}
		\centering
		\includegraphics[width=0.99\textwidth]{plots/alpha/alpha_Udep.pdf}
		\subcaption{
			}
			\label{plot:alpha_Udep}
	\end{subfigure}		
	\hfill
	\centering
	\begin{subfigure}[b]{0.49\textwidth}
		\centering
		\includegraphics[width=0.99\textwidth]{plots/alpha/alpha_600V.pdf}
		\subcaption{
			}
			\label{plot:alpha_600}
	\end{subfigure}
	\hfill
	\begin{subfigure}[b]{0.49\textwidth}
		\centering
		\includegraphics[width=0.99\textwidth]{plots/alpha/alpha_800V.pdf}
		\subcaption{
			}
			\label{plot:alpha_800}
	\end{subfigure}
	\caption{
	    Volume-normalised per-pad leakage current for different fluences at the estimated depletion voltage (a), at a bias voltage of \SI{600}{\volt} (b) and of \SI{800}{\volt} (c).
		Leakage currents were measured after at least \SI{80}{\minute} of additional annnealing at \SI{60}{\celsius}, and were scaled to a temperature of \SI{-20}{\celsius}.
        The current-relate damage rate ($\alpha$) is independent of the sensor production parameters investigated in this work.
	}
\end{figure}

\subsection{Depletion Voltage}

\begin{figure}
	\captionsetup[subfigure]{aboveskip=-1pt,belowskip=-1pt}
	\centering
	\begin{subfigure}[b]{0.49\textwidth}
		\centering
		\includegraphics[width=0.999\textwidth]{plots/Vdep_vs_fluence/Vdep_vs_current_5414.pdf}
		\subcaption{
			}
			\label{plot:Vdep_vs_current_5414}
	\end{subfigure}
	\hfill
	\begin{subfigure}[b]{0.49\textwidth}
		\centering
		\includegraphics[width=0.999\textwidth]{plots/Vdep_vs_fluence/Vdep_vs_current_1002.pdf}
		\subcaption{
		}
		\label{plot:Vdep_vs_current_1002}
	\end{subfigure}	
	\caption{
		Per-pad depletion voltage estimates vs. per-pad leakage current as proxy for the fluence.
	}
\end{figure}

\label{subsec:irradiation_Vdep}

\section{Conclusion}
\label{sec:conclusion}
The CMS High Granularity Calorimeter (HGCAL) is the new proposed granular sampling calorimeter which will replace the existing CMS endcap calorimeters for High-Luminosity LHC (HL-LHC).
For its compactness, its intrinsic timing resolution, and its radiation hardness, p-doped silicon will be deployed as sensitive material in the regions with highest anticipated fluences.
The full silicon sensors are fabricated as 8'' hexagonal wafers with 198 (low density) or 444 (high density design) individual readout pads.
They will have been exposed to fluences ranging from  a few $10^{14}$ to about $10^{16}~\neqcm$ towards the end of HL-LHC's 10-year operation.%\newline
An essential component of the sensors' prototyping is the experimental verification of their radiation hardness in terms of their electrical properties, i.e. leakage currents, capacitances and depletion voltages.
The Rhode Island Nuclear Science Center (RINSC) is one of the few locations world-wide that offers the infrastructure for the irradiation of large structures such as HGCAL's 8''-wide silicon pad sensors.
Given that parts of the necessary infrastructure have not been used in this experimental capacity, the irradiation itself was non-trivial and specific preparations had to be made.
After irradiation, the probe- and switch card-based ARRAY system~\cite{pitters:array2019} was used for the first time for the electrical characterisation of neutron-irradiated silicon sensors at CERN and at Texas Tech University.
In order to protect the test system from currents that would exceed its specifications, the neutron-irradiated sensors were cooled down to \SI{-40}{\celsius} during the tests.%\newline
This paper explained the preparation and execution of the neutron-irradiation at RINSC and the electrical characterisation of the irradiated sensors at cold temperatures using the ARRAY system.
Evidence for non-trivial properties of the RINSC irradiation facility such as a non-uniform flux profile across the 8'' wafer and for non-negligible sensor annealing due to insufficient cooling during irradiation was reported.
We consider the procedural documentation on the neutron-irradiation and on the subsequent electrical characterisation relevant input for future R$\&$D on silicon-based detector concepts for potential future colliders.%\newline
For the planned CMS HGCAL upgrade, the findings in this work are encouraging as they confirm the overall expected radiation hardness of the prototype silicon sensors.
In particular, their leakage current densities are found to scale proportionally with the fluence, independent on the properties of the tested production process variations.
We reconfirm that annealing the sensors up to \SI{80}{\minute} at \SI{60}{\celsius} has beneficial impact on the electrical properties by lowering the dark current and the depletion voltages by a few 10$~\%$.
Despite the tested sensors were prototypes, the insights from their characterisation can be considered representative for the final version.
Therefore, the results reported in this work may serve as reference for the expected electrical performance and degradation of HGCAL's silicon sensors towards the end of their lifetime at HL-LHC.
\acknowledgments

So many people to thank. 


\appendix
\section{Characterisation of the Chuck Temperature Profile}
\label{appendix:chuck_temp}

\bibliographystyle{plain}
\bibliography{bib/bib}

\end{document}
